%%%% SELECT ONE OF THE FOLLOWING COMMANDS %%%%%%%%

%%% TEMPLATE FOR PROCEEDINGS TRACK %%%%
\documentclass[mlabstract]{jmlr}

%% TEMPLATE FOR Extended Abstract Track %%%%%%%
% \documentclass[mlabstract]{jmlr}

%%%%%%%%%%%%%%%%%%%%%%%%%%%%%%%%%%%%%%%%%%%%%%%%%

%%%%%%%%%%%%%%%%%%%%%%%%
% Watermark 
%These 4 commands must be removed for the camera-ready version.
\usepackage[hpos=300px,vpos=70px]{draftwatermark}
\SetWatermarkText{\test}
\SetWatermarkScale{1}
\SetWatermarkAngle{0}
%%%%%%%%%%%%%%%%%%%%%%%%%%


   
% The following packages will be automatically loaded:
% amsmath, amssymb, natbib, graphicx, url, algorithm2e


%%% WARNING %%%%
%%% 1) Please, use the packages automatically loaded to manage references, write equations, and include figures and algorithms. The use of different packages could create problems in the generation of the camera-ready version. Please, follow the examples provided in this file.
%%% 2) References must be included in a .bib file.
%%% 3) Write your paper in a single .tex file.
%%%

%%%% SOFTWARE %%%%
%%% Many papers have associated code provided. If that is your case, include a link to the code in the paper as usual and provide a link to the code in the following comment too. We will use the link in the next comment when we generate the proceedings.
%%% Link to code: http://?? (only for camera ready)

 %\usepackage{rotating}% for sideways figures and tables
\usepackage{longtable}% for long tables

 % The booktabs package is used by this sample document
 % (it provides \toprule, \midrule and \bottomrule).
 % Remove the next line if you don't require it.
\usepackage{booktabs}
 % The siunitx package is used by this sample document
 % to align numbers in a column by their decimal point.
 % Remove the next line if you don't require it.
\usepackage[load-configurations=version-1]{siunitx} % newer version
 %\usepackage{siunitx}

 % The following command is just for this sample document:
\newcommand{\cs}[1]{\texttt{\char`\\#1}}

 % Define an unnumbered theorem just for this sample document:
\theorembodyfont{\upshape}
\theoremheaderfont{\scshape}
\theorempostheader{:}
\theoremsep{\newline}
\newtheorem*{note}{Note}

%%%% DON'T CHANGE %%%%%%%%%
\jmlrvolume{}
\firstpageno{1}
\editors{List of editors' names}

\jmlryear{2023}
\jmlrworkshop{Symmetry and Geometry in Neural Representations}

%\editor{Editor's name}
%%%%%%%%%%%%%%%%%%%%%%%%%%%



\title[Equivariant Planning]{Equivariant World Models for Planning}

%%%%%%%%%%%%%%%%%%%%%%%%%%%%%%%%%%%%%
% THE MANUSCRIPT, DATA AND CODE MUST BE ANONYMIZED DURING THE REVIEW PROCESS. 
% DON'T INCLUDE ANY INFORMATION ABOUT AUTHORS DURING THE REVIEW PROCESS.
% Information about authors (Full names, emails, affiliations) have to be provided only for the submission of the camera-ready version.  Only in that case, you can uncomment and use the next blocks.
%%%%%%%%%%%%%%%%%%%%%%%%%%%%%%%%%%%%%

 % Use \Name{Author Name} to specify the name.

 % Spaces are used to separate forenames from the surname so that
 % the surnames can be picked up for the page header and copyright footer.
 
 % If the surname contains spaces, enclose the surname
 % in braces, e.g. \Name{John {Smith Jones}} similarly
 % if the name has a "von" part, e.g \Name{Jane {de Winter}}.
 % If the first letter in the forenames is a diacritic
 % enclose the diacritic in braces, e.g. \Name{{\'E}louise Smith}

 % *** Make sure there's no spurious space before \nametag ***

 % Two authors with the same address
%   \author{\Name{Author Name1\nametag{\thanks{with a note}}} \Email{abc@sample.com}\and
%   \Name{Author Name2} \Email{xyz@sample.com}\\
%   \addr Address}

  %Three or more authors with the same address:
%   \author{\Name{Author Name1} \Email{an1@sample.com}\\
%   \Name{Author Name2} \Email{an2@sample.com}\\
%   \Name{Author Name3} \Email{an3@sample.com}\\
%   \Name{Author Name4} \Email{an4@sample.com}\\
%   \Name{Author Name5} \Email{an5@sample.com}\\
%   \Name{Author Name6} \Email{an6@sample.com}\\
%   \Name{Author Name7} \Email{an7@sample.com}\\
%   \Name{Author Name8} \Email{an8@sample.com}\\
%   \Name{Author Name9} \Email{an9@sample.com}\\
%   \Name{Author Name10} \Email{an10@sample.com}\\
%   \Name{Author Name11} \Email{an11@sample.com}\\
%   \Name{Author Name12} \Email{an12@sample.com}\\
%   \Name{Author Name13} \Email{an13@sample.com}\\
%   \Name{Author Name14} \Email{an14@sample.com}\\
%   \addr Address}


 % Authors with different addresses:
 \author{\Name{Sean Craven} \Email{sean.craven.22@ucl.ac.uk, sean.craven@advai.co.uk}\\
 \AND
\Name{Caswell Barry} \Email{}\\}
 % \addr Address 2
 %}



\begin{document}

\maketitle

\begin{abstract}
	This is the abstract for this article.
\end{abstract}

\section{Introduction}
Encoding equivariance into agent networks has been an effective inductive bias for Reinforcement learning agents in symmetric environments. Enabling agents to learn effective policies with superior sample efficiency \cite{van2020plannable, mondal2020group} or learn policies on previously unapproachable tasks \cite{wang2022so2}. These works exploit properties of Group structured MDP homomorphisms~\cite{ravindran2003smdp, ravindran2001symmetries}. Which describes an abstract MDP, where in the deterministic case:
\begin{equation}
  T(S, A) = S', \\
  T(g \cdot S, g \cdot a) = g \cdot S'.
  \label{eq:gs_mdp}
\end{equation}
\begin{equation}
  R(S, A) = r, \\
  R(g \cdot S, g \cdot A) = r.
  \label{eq:gs_mdp_rw}
\end{equation}
Here $S, S' \in \mathcal{S}$ are states, and $A \in \mathcal{A}$ are actions. $g \in G$ are group actions acting on the state action space. This definition is taken from \cite{van2020plannable}, with slightly altered notation. These notions can be extended to stochastic MDPs.

Our preliminary findings focus on extending the equivariant inductive bias to a transition model of the environment, to enable planning. Our initial investigation focus around augmenting a PPO agent with the ability to plan using learned transition models.

We implemented a Dyna-style agent which directly learning a policy from both the real environment, and trajectories simulated by a equivariant transition model. To achieve this a novel method of polling is proposed to produce approximately equivariant transition models. In our initial testing we see promising potential when planning with equivariant transition models. However, we failed to implement a Dyna-style agent that outperformed a PPO agent in the Cart-Pole and Catch environments that were tested in.

\section{Method}
The Dyna-style training process for the equivariant agents consists of equal length planning and acting phases. In the acting phase the agent gains experience in the real environment, which is followed by a planning phase. Where the agent gains experience from a simulated MDP. The transition model, $T_\phi$, of the simulated MDP is constructed with a Group-Convolutional Neural Network (G-CNN)~\cite{cohen2016group}. 

To construct a transition model $T: \mathcal{S} \times \mathcal{A} \rightarrow \mathcal{A}$, maps from state and action space to state space. Where each state has a well defined shape.

Each group convolution layer is constructed out of a single kernel which is convolved with the input. For each group action in a given symmetry group that the layer is equivariant to the kernel is operated on by a group action. Thus a layer takes the form of
\begin{equation}
  \begin{pmatrix}
     k \cdot g_0* f \\
     k \cdot g_1 * f\\
     \vdots \\
     k \cdot g_{|G|} * f
  \end{pmatrix}.
  \label{eq:g-cnn}
\end{equation}
Where $k$ is the kernel cross-correlated with the input $f$. In our setting each cross-correlation produces an output of the same shape as a state. As such, a reduction must be performed to get an output in state space. If a max or average pooling was used over the $|G|$ dimension the equivariance of the group convolution would be lost. The network would become invariant to group operations on the input. This loss of equivariance motivates the need for an equivariant reduction to ensure that the transition models used for planning are fully equivariant.

The proximal pooling layer provides a novel and empirically reliable method to produce equivariant networks. Inspired by the intuition that transitions to subsequent states in many environments are closer to the previous state than a random state sampled from the environment. It only requires a distance metric, $d(s, s')$ that measures the difference between two states. In both Catch and Cart-Pole the L1 distance is used between parts of the state.

The forward pass of the network outputs multiple 



\label{sec:intro}

\acks{Acknowledgements go here.}

\bibliography{pmlr-sample}

\appendix


\section{Second Appendix}\label{apd:second}


\end{document}
