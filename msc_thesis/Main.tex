% UCL Thesis LaTeX Template
%  (c) Ian Kirker, 2014
% 
% This is a template/skeleton for PhD/MPhil/MRes theses.
%
% It uses a rather split-up file structure because this tends to
%  work well for large, complex documents.
% We suggest using one file per chapter, but you may wish to use more
%  or fewer separate files than that.
% We've also separated out various bits of configuration into their
%  own files, to keep everything neat.
% Note that the \input command just streams in whatever file you give
%  it, while the \include command adds a page break, and does some
%  extra organisation to make compilation faster. Note that you can't
%  use \include inside an \include-d file.
% We suggest using \input for settings and configuration files that
%  you always want to use, and \include for each section of content.
% If you do that, it also means you can use the \includeonly statement
%  to only compile up the section you're currently interested in.
% You might also want to put figures into their own files to be \input.

% For more information on \input and \include, see:
%  http://tex.stackexchange.com/questions/246/when-should-i-use-input-vs-include


% Formatting rules for theses are here: 
%  http://www.ucl.ac.uk/current-students/research_degrees/thesis_formatting
% Binding and submitting guidelines are here:
%  http://www.ucl.ac.uk/current-students/research_degrees/thesis_binding_submission

% This package goes first and foremost, because it checks all 
%  your syntax for mistakes and some old-fashioned LaTeX commands.
% Note that normally you should load your documentclass before 
%  packages, because some packages change behaviour based on
%  your document settings.
% Also, for those confused by the RequirePackage here vs usepackage
%  elsewhere, usepackage cannot be used before the documentclass
%  command, while RequirePackage can. That's the only functional
%  difference as far as I'm aware.
\RequirePackage[l2tabu, orthodox]{nag}


% ------ Main document class specification ------
% The draft option here prevents images being inserted,
%  and adds chunky black bars to boxes that are exceeding 
%  the page width (to show that they are).
% The oneside option can optionally be replaced by twoside if
%  you intend to print double-sided. Note that this is
%  *specifically permitted* by the UCL thesis formatting
%  guidelines.
%
% Valid options in terms of type are:
%  phd
%  mres
%  mphil
%\documentclass[12pt,phd,draft,a4paper,oneside]{ucl_thesis}
\documentclass[12pt,phd,a4paper,oneside]{ucl_thesis}


% Package configuration:
%  LaTeX uses "packages" to add extra commands and features.
%  There are quite a few useful ones, so we've put them in a 
%   separate file.
% -------- Packages --------

% This package just gives you a quick way to dump in some sample text.
% You can remove it -- it's just here for the examples.
\usepackage{blindtext}

% This package means empty pages (pages with no text) won't get stuff
%  like chapter names at the top of the page. It's mostly cosmetic.
\usepackage{emptypage}

% The graphicx package adds the \includegraphics command,
%  which is your basic command for adding a picture.
\usepackage{graphicx}

% The float package improves LaTeX's handling of floats,
%  and also adds the option to *force* LaTeX to put the float
%  HERE, with the [H] option to the float environment.
\usepackage{float}

% The amsmath package enhances the various ways of including
%  maths, including adding the align environment for aligned
%  equations.
\usepackage{amsmath}
\usepackage{amssymb}

% Use these two packages together -- they define symbols
%  for e.g. units that you can use in both text and math mode.
% \usepackage{gensymb}
% \usepackage{textcomp}
% You may also want the units package for making little
%  fractions for unit specifications.
%\usepackage{units}


% The setspace package lets you use 1.5-sized or double line spacing.
\usepackage{setspace}
\setstretch{1.5}

% That just does body text -- if you want to expand *everything*,
%  including footnotes and tables, use this instead:
%\renewcommand{\baselinestretch}{1.5}


% PGFPlots is either a really clunky or really good way to add graphs
%  into your document, depending on your point of view.
% There's waaaaay too much information on using this to cover here,
%  so, you might want to start here:
%   http://pgfplots.sourceforge.net/
%  or here:
%   http://pgfplots.sourceforge.net/pgfplots.pdf
%\usepackage{pgfplots}
%\pgfplotsset{compat=1.3} % <- this fixed axis labels in the version I was using

% PGFPlotsTable can help you make tables a little more easily than
%  usual in LaTeX.
% If you're going to have to paste data in a lot, I'd suggest using it.
%  You might want to start with the manual, here:
%  http://pgfplots.sourceforge.net/pgfplotstable.pdf
%\usepackage{pgfplotstable}

% These settings are also recommended for using with pgfplotstable.
%\pgfplotstableset{
%	% these columns/<colname>/.style={<options>} things define a style
%	% which applies to <colname> only.
%	empty cells with={--}, % replace empty cells with '--'
%	every head row/.style={before row=\toprule,after row=\midrule},
%	every last row/.style={after row=\bottomrule}
%}


% The mhchem package provides chemistry formula typesetting commands
%  e.g. \ce{H2O}
%\usepackage[version=3]{mhchem}

% And the chemfig package gives a weird command for adding Lewis 
%  diagrams, for e.g. organic molecules
%\usepackage{chemfig}

% The linenumbers command from the lineno package adds line numbers
%  alongside your text that can be useful for discussing edits 
%  in drafts.
% Remove or comment out the command for proper versions.
%\usepackage[modulo]{lineno}
% \linenumbers 


% Alternatively, you can use the ifdraft package to let you add
%  commands that will only be used in draft versions
%\usepackage{ifdraft}

% For example, the following adds a watermark if the draft mode is on.
%\ifdraft{
%  \usepackage{draftwatermark}
%  \SetWatermarkText{\shortstack{\textsc{Draft Mode}\\ \strut \\ \strut \\ \strut}}
%  \SetWatermarkScale{0.5}
%  \SetWatermarkAngle{90}
%}


% The multirow package adds the option to make cells span 
%  rows in tables.
\usepackage{multirow}


% Subfig allows you to create figures within figures, to, for example,
%  make a single figure with 4 individually labeled and referenceable
%  sub-figures.
% It's quite fiddly to use, so check the documentation.
%\usepackage{subfig}

% The natbib package allows book-type citations commonly used in
%  longer works, and less commonly in science articles (IME).
% e.g. (Saucer et al., 1993) rather than [1]
% More details are here: http://merkel.zoneo.net/Latex/natbib.php
%\usepackage{natbib}

% The bibentry package (along with the \nobibliography* command)
%  allows putting full reference lines inline.
%  See: 
%   http://tex.stackexchange.com/questions/2905/how-can-i-list-references-from-bibtex-file-in-line-with-commentary
\usepackage{bibentry}

% The isorot package allows you to put things sideways 
%  (or indeed, at any angle) on a page.
% This can be useful for wide graphs or other figures.
%\usepackage{isorot}

% The caption package adds more options for caption formatting.
% This set-up makes hanging labels, makes the caption text smaller
%  than the body text, and makes the label bold.
% Highly recommended.
\usepackage[format=hang,font=small,labelfont=bf]{caption}

% If you're getting into defining your own commands, you might want
%  to check out the etoolbox package -- it defines a few commands
%  that can make it easier to make commands robust.
\usepackage{etoolbox}


% Sets up links within your document, for e.g. contents page entries
%  and references, and also PDF metadata.
% You should edit this!
%%
%% This file uses the hyperref package to make your thesis have metadata embedded in the PDF, 
%%  and also adds links to be able to click on references and contents page entries to go to 
%%  the pages.
%%

% Some hacks are necessary to make bibentry and hyperref play nicely.
% See: http://tex.stackexchange.com/questions/65348/clash-between-bibentry-and-hyperref-with-bibstyle-elsart-harv
\usepackage{bibentry}
\makeatletter\let\saved@bibitem\@bibitem\makeatother
\usepackage[pdftex,hidelinks]{hyperref}
\makeatletter\let\@bibitem\saved@bibitem\makeatother
\makeatletter
\AtBeginDocument{
    \hypersetup{
        pdfsubject={Thesis Subject},
        pdfkeywords={Thesis Keywords},
        pdfauthor={Author},
        pdftitle={Title},
    }
}
\makeatother
    


% And then some settings in separate files.
% These settings are from:
%  http://mintaka.sdsu.edu/GF/bibliog/latex/floats.html

% They give LaTeX more options on where to put your figures, and may
%  mean that fewer of your figures end up at the tops of pages far
%  away from the thing they're related to.

% Alters some LaTeX defaults for better treatment of figures:
% See p.105 of "TeX Unbound" for suggested values.
% See pp. 199-200 of Lamport's "LaTeX" book for details.

%   General parameters, for ALL pages:
\renewcommand{\topfraction}{0.9}	% max fraction of floats at top
\renewcommand{\bottomfraction}{0.8}	% max fraction of floats at bottom

%   Parameters for TEXT pages (not float pages):
\setcounter{topnumber}{2}
\setcounter{bottomnumber}{2}
\setcounter{totalnumber}{4}     % 2 may work better
\setcounter{dbltopnumber}{2}    % for 2-column pages
\renewcommand{\dbltopfraction}{0.9}	% fit big float above 2-col. text
\renewcommand{\textfraction}{0.07}	% allow minimal text w. figs

%   Parameters for FLOAT pages (not text pages):
\renewcommand{\floatpagefraction}{0.7}	% require fuller float pages
% N.B.: floatpagefraction MUST be less than topfraction !!
\renewcommand{\dblfloatpagefraction}{0.7}	% require fuller float pages

% remember to use [htp] or [htpb] for placement,
% e.g. 
%  \begin{figure}[htp]
%   ...
%  \end{figure} % For things like figures and tables
\bibliographystyle{apalike}

   % For bibliographies

% These control how many number sections your subsections will take
%    e.g. Section 2.3.1.5.6.3
%  and how many of those will get put into the contents pages.
\setcounter{secnumdepth}{3}
\setcounter{tocdepth}{3}


\begin{document}

% ^-- This is a dumb trick that works with the bibentry package to let
%  you put bibliography entries whereever you like.
% I used this to put references to papers a chapter's work was 
%  published in at the end of that chapter.
% For more information, see: http://stefaanlippens.net/bibentry

% If you haven't finished making your full BibTex file yet, you
%  might find this useful -- it'll just replace all your
%  citations with little superscript notes.
% Uncomment to use.
%\renewcommand{\cite}[1]{\emph{\textsuperscript{[#1]}}}

% At last, content! Remember filenames are case-sensitive and 
%  *must not* include spaces.
\newcommand{\mc}[1]{\mathcal{\#1}}
\newcommand{\omc}[1]{\overline{\mathcal{#1}}}
% I may change the way this is done in a future version, 
%  but given that some people needed it, if you need a different degree title 
%  (e.g. Master of Science, Master in Science, Master of Arts, etc)
%  uncomment the following 3 lines and set as appropriate (this *has* to be before \maketitle)
% \makeatletter
% \renewcommand {\@degree@string} {Master of Things}
% \makeatother

\title{Exploiting Symmetries With Equivariant Transition Models}
\author{YBHQ1\\Supervisor: Prof. Caswell Barry}
\department{Computer Science}

\maketitle

\begin{abstract} % 300 word limit
	This report uses Group Convolutions~\cite{cohen2016group} to construct Equivariant networks used to parameterize reinforcement learning agents, which act in symmetric environments. The Group Convolutions enable the symmetries of the environments to be encoded into the networks, enabling superior generalization. On investigation, using equivariant networks provides substantially better sample efficiency for the reinforcement learning agents. When applied to a model-based setting, equivariant transition models demonstrated superior sample efficiency in learning and outperformed conventional multi-layer perceptron architectures on the symmetric environments tested. However, in the experiments with the Dyna model-based algorithm learning transition models online, was unsuccessful.
\end{abstract}


\setcounter{tocdepth}{2}
% Setting this higher means you get contents entries for
%  more minor section headers.

\tableofcontents
\listoffigures
\listoftables


\chapter{Introduction}\label{Chap1}
The power of symmetry in understanding the physical world has been surprisingly effective. Perhaps one of the most famous examples of this is the eightfold way of Gell-Mann\cite{gellmann1961eight}, which brought much deeper insight into the structure of elementary particles.

The power of symmetry is not only useful in abstract Physics. The natural world has a bias for symmetries~\cite{johnston2022symmetry}, and the simplicity they enable. There also exists evidence that humans exploit symmetries in cognitively challenging tasks~\cite{he2022symmetry}. This apparent bias towards symmetry and evidence that humans leverage it in making decisions  motivates the exploration of artificial decision-making agents with the ability to leverage the idea of symmetry, which this report focuses upon.


Symmetries are a wider concept than that of things with the same reflection down a centre line. Symmetries describe transformations that leave something unchanged, for example rotating a square by $90^o$. However, the transformations are not limited to rotations and reflections, but can extend to wider ideas of time inversion, permutations and translations. Symmetries also enable more abstract mathematical reasoning through the ideas of groups. The more abstract notions of symmetry have enabled more far-reaching results. Such as, Noether's theorem, which suggests that symmetries are a fundamental source of conservation laws, such as energy conservation, momentum, and angular momentum.

In machine learning ideas describing fundamental rules about the world and associated rules are known as inductive biases. In order to make the agent understand these symmetries, there are two main approaches to induce these priori ideas into the agent: Data augmentation, and encoding the invariance into the agent.

Data Augmentation techniques leverage known symmetries in environment and create artificial training data by transforming the input such that the symmetry is respected. For example, flipping an input image horizontally~\cite{laskin2020reinforcement, yijion2020invariant}. Such techniques have substantial history in supervised learning, where it is common to increase the amount of data by performing transformations on the input that preserve the output.

The second approach is to structure the agents' neural networks (NN) to only learn behaviours that respect the environment's symmetry\cite{vanderpol2020mdp,wang2022so2, mondal2020group}. Notably, addressing the issue of generalizing to symmetries represented a key breakthrough in deep learning with the introduction of Convolutional Networks (CNNs)\cite{lecun1989backprop}. These networks are translationally invariant, which is particularly important in computer vision. Further enhancements, like Group-Convolution\cite{cohen2016group}, extend this equivariant behaviour to accommodate a wider range of symmetries.

This report investigates the use of Group-Convolutional neural networks to introduce symmetric inductive biases into decision-making agents. These agents operate within the paradigm of Reinforcement Learning (RL), in which the agent gains and learns from experience to improve its decision-making capabilities. Additionally, the report explores incorporating symmetric inductive biases into a planning phase. In this phase, rather than directly learning from experience in solving a problem, the agent 'imagines' the results of its actions. This approach aims to enhance the agent's problem-solving and decision-making capabilities.

\chapter{My First Content Chapter}
\label{chapterlabel2}

% This just dumps some pseudolatin in so you can see some text in place.
\blindtext

\section{Notes}

\textbf{Definition: MDP Homomorphism} Given some MDP $\mathcal{M}$, where there
exists a surjective map, from $\mathcal{S} \times \mathcal{A} \rightarrow \omc{S} \times
	\omc{A}$. the MDP $\omc{M}$ is an abstract MDP over the new space. The
homomorphism $h$, then is the tuple of $(\sigma, \alpha_s|s \in \mathcal{S})$, where
$\sigma: \mathcal{S} \rightarrow \omc{S}$ and $\alpha_s : \mathcal{A} \rightarrow \omc{A}$.
This surjective map must satisfy two constraints for it to be a valid MDP
homomorphism; \begin{enumerate} \item $R(s, a) = R(\sigma(s), \alpha_s(a))$
	\item $T(s', a, s) = T(\sigma(s'), \alpha_s(a), \sigma(s))$ \end{enumerate}


Rather than learning a tranditional MDP homomorphism, we wish to learn a homomorphic map

$$h: \mathcal{A} \times \mathcal{S} \times \overline{\mathcal{A}} -> \omc{S}$$

With the constraints that the homomorphsim in the simplest case maps of invertersion symmetry $D_2$ where in state space the $D_2$ representtion is $L_2$ and in action space the $D_2$ representaiton is $K_2$. so that $$h(a, s, \overline{a}) = h(K_2 a, L_2 s, \overline{a})$$.

Because we are learning from determinsistic dynamics $T: \mathcal{S} \times \mathcal{A} \rightarrow \mathcal{S}$ and $\overline{T}: \omc{S} \times \omc{A} \rightarrow \omc{S}$ we must also learn that

$$\overline{T}(\overline{s}, \overline{a}) = h(T(s, a), a', \overline{a})$$.

where $overline{s} = h(s, a, \overline{a})$

$$\overline{T}(h(s, a, \overline{a}), \overline{a}) = h(T(s, a), a', \overline{a})$$.


In the case of the cartpole the actions $a, \overline{a} \in [0, 1]$.


\subsection{Training}
The objectiive of training is to use some kind of similarity metric like that used in Approximate MDP Homomorphisms to learn the parametric form of $h$ in a supervised setting, over these (state, action, abstract action, next state) tuples.
With the hypothesis that having a group equivariant network will make the learning more efficient.

\chapter{General Conclusions}
\label{chapterlabel4}

% This just dumps some pseudolatin in so you can see some text in place.
\blindtext

\addcontentsline{toc}{chapter}{Appendices}

% The \appendix command resets the chapter counter, and changes the chapter numbering scheme to capital letters.
%\chapter{Appendices}
\appendix
\chapter{An Appendix About Stuff}
\label{appendixlabel1}
(stuff)

\chapter{Another Appendix About Things}
\label{appendixlabel2}
(things)

\chapter{Colophon}
\label{appendixlabel3}
\textit{This is a description of the tools you used to make your thesis. It helps people make future documents, reminds you, and looks good.}

\textit{(example)} This document was set in the Times Roman typeface using \LaTeX\ and Bib\TeX , composed with a text editor. 
 % description of document, e.g. type faces, TeX used, TeXmaker, packages and things used for figures. Like a computational details section.
% e.g. http://tex.stackexchange.com/questions/63468/what-is-best-way-to-mention-that-a-document-has-been-typeset-with-tex#63503

% Side note:
%http://tex.stackexchange.com/questions/1319/showcase-of-beautiful-typography-done-in-tex-friends

% You could separate these out into different files if you have
%  particularly large appendices.

% This line manually adds the Bibliography to the table of contents.
% The fact that \include is the last thing before this ensures that it
% is on a clear page, and adding it like this means that it doesn't
% get a chapter or appendix number.
\addcontentsline{toc}{chapter}{Bibliography}
% Actually generates your bibliography.

\bibliography{./example}
% All done. \o/
\end{document}
