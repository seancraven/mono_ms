\chapter{Introduction}\label{Chap1}
The power of symmetry in understanding the physical world has been surprisingly effective. Perhaps one of the most famous examples of this is the eightfold way of Gell-Mann\cite{gellmann1961eight}, which brought much deeper insight into the structure of elementary particles.

The power of symmetry is not only useful in abstract Physics. The natural world has a bias for symmetries~\cite{johnston2022symmetry}, and the simplicity they enable. There also exists evidence that humans exploit symmetries in cognitively challenging tasks~\cite{he2022symmetry}. This apparent bias towards symmetry and evidence that humans leverage it in making decisions  motivates the exploration of artificial decision-making agents with the ability to leverage the idea of symmetry, which this report focuses upon.


Symmetries are a wider concept than that of things with the same reflection down a centre line. Symmetries describe transformations that leave something unchanged, for example rotating a square by $90^o$. However, the transformations are not limited to rotations and reflections, but can extend to wider ideas of time inversion, permutations and translations. Symmetries also enable more abstract mathematical reasoning through the ideas of groups. The more abstract notions of symmetry have enabled more far-reaching results. Such as, Noether's theorem, which suggests that symmetries are a fundamental source of conservation laws, such as energy conservation, momentum, and angular momentum.

In machine learning ideas describing fundamental rules about the world and associated rules are known as inductive biases. In order to make the agent understand these symmetries, there are two main approaches to induce these priori ideas into the agent: Data augmentation, and encoding the invariance into the agent.

Data Augmentation techniques leverage known symmetries in environment and create artificial training data by transforming the input such that the symmetry is respected. For example, flipping an input image horizontally~\cite{laskin2020reinforcement, yijion2020invariant}. Such techniques have substantial history in supervised learning, where it is common to increase the amount of data by performing transformations on the input that preserve the output.

The second approach is to structure the agents' neural networks (NN) to only learn behaviours that respect the environment's symmetry\cite{vanderpol2020mdp,wang2022so2, mondal2020group}. Notably, addressing the issue of generalizing to symmetries represented a key breakthrough in deep learning with the introduction of Convolutional Networks (CNNs)\cite{lecun1989backprop}. These networks are translationally invariant, which is particularly important in computer vision. Further enhancements, like Group-Convolution\cite{cohen2016group}, extend this equivariant behaviour to accommodate a wider range of symmetries.

This report investigates the use of Group-Convolutional neural networks to introduce symmetric inductive biases into decision-making agents. These agents operate within the paradigm of Reinforcement Learning (RL), in which the agent gains and learns from experience to improve its decision-making capabilities. Additionally, the report explores incorporating symmetric inductive biases into a planning phase. In this phase, rather than directly learning from experience in solving a problem, the agent 'imagines' the results of its actions. This approach aims to enhance the agent's problem-solving and decision-making capabilities.
